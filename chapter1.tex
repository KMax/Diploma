\chapter{Описание предметной области}
\section{Проблемы и задачи экспертных систем}

Цель любой экспертной системы (ЭС) состоит в решении задач без прямого участия эксперта в предметной области. Принципиально отличие ЭС от других программ является то что она не просто ассистирует человеку, выполняя часть работы, а консультирует в какой-либо конкретной предметной области. ЭС не призваны заменить собою эксперта, в него непосредственной деятельности, а на наоборот расширяют возможную сферу применения знаний авторитетных специалистов, за счет аккумулирования и тиражирования опыта и знаний высококвалифицированных специалистов, позволяя пользоваться этими знаниями пользователям, не являющимися специалистами в данной предметной области. Кроме того экспертные системы имеют некоторые преимущества над человеком-экспертом:
\begin{labelitemi}
	\labelitemi Доступность: ЭС может работать там, где есть компьютер,
	\labelitemi Стоимость: цена для отдельного пользователя гораздо ниже,
	\labelitemi Опасность: может быть использована в условиях не пригодных для деятельности человека,
	\labelitemi Срок эксплуатации: зависит только от оборудования,
	\labelitemi Скорость: может предоставлять быстрый ответ, либо в режиме реального времени для критичных приложений,
	\labelitemi Достоверность: может предоставлять мнение, отличное от мнения человека-эксперта.
\end{labelitemi}
Но и имеют ограничения, которые не присущи эксперту:

    Отсутствие глубоких знаний. ЭС не имеет понятия о реальных причинах или следствиях в системе, в основном это потому что легче запрограммировать поверхностные знания, основанные на эмпирических и эвристических знаниях. А проектирование ЭС основанной на базовых понятиях и поведении некоторых базовых объектов потребует намного больше усилий, а в результате система будет слишком сложной для поддержки.

    Незнание границ неопределенности. Человек-эксперт знает границы своих знаний и может изменить свою рекомендацию, когда проблема требует знаний за пределами знаний эксперта. Когда как ЭС предложит рекомендацию даже в ситуации, когда данных не хватает для предоставления реального решения.

    Существующие ЭС не могут проводить аналогии, например обобщать имеющиеся знания, для того чтобы рассуждать о новой ситуации так как это может человек.

В настоящее время экспертные системы используется для решения различных типов задач (интерпретация, предсказание, диагностика, планирование, конструирование, контроль, отладка, инструктаж, управление) в самых разнообразных проблемных областях, таких, как финансы,  нефтяная и газовая промышленность, энергетика, транспорт, фармацевтическое производство, космос, металлургия, горное дело, химия, образование, целлюлозно–бумажная промышленность, телекоммуникации и связь и др.