\intro
Экспертные системы (ЭС) были разработаны как научно-исследовательские инструментальные средства в 1960-х годах. Первое коммерческое внедрение произошло в 1980-х годах и с того времени они получили широкое распространение. Их предназначением является тиражирование опыта и знаний опытных экспертов в различных предметных областях, позволяющее устранить нехватку специалистов-экспертов, которые смог ли бы отвечать на многочисленные вопросы в своей области знаний. ЭС являются одним из успешных направлений искусственного интеллекта, применяемое в различных процессах принятия решений и решения задач в разных сферах социальной и технической деятельности человека.

Одной из таких сфер применения является поддержка систем автоматизированного проектирования (САПР). Здесь экспертные системы позволяют усилить автоматизацию традиционных САПР за счёт встраивания правил проектирования и инженерных знаний об оптимизации процессов проектирования, обеспечивающих решения различных задач, как по автоматизация повторяющихся задач, не требующих “творческого” мышления так и задач требующих мультидисциплинарных знаний.

Данная работа посвящена разработке продукционной экспертной системы нацеленной на автоматизацию одного из этапов проектирования оптических систем (ОС) - структурного синтеза. Так же работа ограничена только одним классом оптических систем - фотообъективами (в дальнейшем приставка “фото” будет опущена).

Работа является продолжением реализации теории композиции М.М. Русинова [1] и её развития в работах И.Л.Лившиц [2][3]. Потенциал данной теории и сегодня остается раскрытым не полностью, что подтверждается интересом со стороны международного сообщества к данной тематике. Основными задачами теории композиции ОС является классификация элементов в оптической системе и анализ их применимости в тех или иных случаях. Элементы оптической системы по своему назначению разделяются на базовые (B), коррекционные (C), светосильные (T) и широкоугольные (Y). Получение описания последовательности элементов для достижения конкретных оптических характеристик называется структурным синтезом и является первым этапом автоматизированного проектирования ОС.

Задача структурного синтеза оптической системы не имеет детерминированного алгоритма решения, так как под одни и те же техническим требованиям может подходить большое количество как похожих, так и абсолютно различных структурных схем. Поэтому специалист-оптик может полагаться только на свой опыт и знания в проектировании ОС в выборе оптимальной структурной схемы. И чем он опытнее, тем более оптимальная будет выбрана схема.

Теория Русинова и Лившиц, описывает огромный опыт и знания в проектировании ОС, который способствует его формализации и использовании в качестве основы для разработки экспертной системы автоматизирующей процесс выбора структурной схемы ОС.